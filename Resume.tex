\documentclass[8pt]{developercv}

\newcommand{\CC}{C\nolinebreak\hspace{-.05em}\raisebox{.4ex}{\tiny\bf +}\nolinebreak\hspace{-.10em}\raisebox{.4ex}{\tiny\bf +}}
\def\CC{{C\nolinebreak[4]\hspace{-.05em}\raisebox{.4ex}{\tiny\bf ++}}}

\begin{document}
	\begin{minipage}[a]{1.0\textwidth}
		\begin{center}	%header%
			{\HUGE Resume}\\
			\textbf {Hudson Cassidy} \\
			\textbf {Email} hudoc96@hotmail.com
			\textbf {Phone} 0466651465 \\
			\textbf {Github} https://github.com/CrimsonNynja \\
		\end{center}
	\end{minipage}
	\begin{minipage}[t]{0.7\textwidth}
		\cvsect{Work Experience}
		\begin{entrylist}	%work experience%
			\entry
				{Cognativ}
				{Developer}
				{2020}
				{Worked as a Developer for Cognativ who do contract work, primarily for mobile applications. My work here primarily consists of node.js in the backend, and react or react native in the front end depending on the project.
				While here, I have worked on:
				\begin{itemize}
					\item A podcast app, the backend of which written in Django, and the front end in React native. This app was inherited, so a lot of effort was placed into helpng code maintainability. As such I made the initiative to start using pydocs, as well as splitting up massive files into appropriate modules. I also stressed that more effort was needed in code readability, to stop the many problems and styles currently in the code base from continuing.
					\item A social media app's backend, written with a Node.js backend. While working in this, I rewrote the backend to be fully tested, with 100\% code coverage, and to utilize better code practices, like environment variables for its secret keys, and separating its graphql resolvers from the its database functions.
					\item The same app's admin website, written in React. I started this one from scratch, so have been in charge of almost all aspects of this, including project management.
				\end{itemize}
				}
			\entry
				{Plezzel}
				{Developer}
				{2018 - 2019}
				{Joined a small start-up in the Real Estate / Digital Marketing space in Geelong as the 3rd developer. While there I took the following initiatives:
				\begin{itemize}
					\item Rewrote the email template engine, cutting development time from 1 week to a day per template. This system needed to allow each template to be styled and branded differently, as well as provide some rudimentary logic. This included separating out the data from the visuals, and providing a layer that ensures that email compatible HTML / CSS is produced.
					\item Learned and used GraphQL and Laravel to rebuild the mobile app's backend from scratch.
					\item Helped the team rewrite old code into new maintainable modules, including starting to write unit tests, and incorporate proper git processes (code reviews, better branch management) and start using Jira to manage sprints.
				\end{itemize}
				}
		\end{entrylist}			
		\cvsect{Education}
		\begin{entrylist}	%education%
			\entry
				{RMIT}
				{Bachelor of Compuer Science (RMIT University)}
				{2015-2017}
				{Completed my Bachelor of Computer Science at RMIT Melbourne. I took my majors in Cloud Computing and Artificial Intelligence }
		\end{entrylist}
		\cvsect{Personal Projects}  %add a reasoning for each of these to exist%
		\begin{entrylist}	%yeoman%
			\entry
				{Node\newline Graphql\newline Scaffolder}
				{https://github.com/CrimsonNynja/generator-node-graphql}
				{}
				{Created a code scaffolder that generates a fresh node.js / graphql project that tries to follow best practices. The generated code comes complete with unit tests, JWT auth, as well as minimal code to create / login to a user using the auth. It tries to be unbiased to semantic decisions where possible. This is also available as an NPM package, which gets around 60 downloads a week}
		\end{entrylist}
		\begin{entrylist}	%php trees%
			\entry
				{PHP Tree\newline Library}
				{https://github.com/CrimsonNynja/PHP-Trees}
				{}
				{Library for PHP that adds tree structures (BST, rope, heap). The trees are designed to seamlessly work within the language, utilizing the latest PHP features. Full unit tests are also provided, written in PHP Unit, and built with PSR in mind. The latest release is also available as a composer package. I built this as PHP does not have any trees structures in its core, and any libraries for them are very sparse, to non existent. }
		\end{entrylist}
	\end{minipage}
\begin{minipage}[t]{0.05\textwidth}
\hphantom{0.1}
\end{minipage}
	\begin{minipage}[t]{0.3\textwidth}		%side bar%
		\cvsect{References}\\
		\textbf{Max Bush} 04 0418 0650\\
		\textbf{Jahryn Galbraith} 04 0905 4508\\

		\cvsect{Current Areas of interest}\\
		{PHP}\\
		{Javascript (React, Node)}\\
		{Python}\\
		{GraphQL}\\
		{Data Structures}\\
		{Test Frameworks}\\
		
		\cvsect{experience}\\
		\\\textbf{Programming Languages}\\
		{Javascript}\\
		{PHP}\\
		{\CC}\\
		{HTML / CSS}\\
		{Python}\\
		{SQL}\\

		\textbf{Tools / Frameworks}\\
		{React}\\
		{Node}\\
		{Git}\\
		{Jira}\\
		{SFML}\\
		{GraphQL}\\
		{React Native}\\
		{AWS}\\
		
		\textbf{JS packages}\\
		{express}\\
		{apollo}\\
		{Jest}\\
		{lodash}\\
		{react}\\
		{react-router}\\
		{styled-components}\\
		{typescript}\\
		{axios}\\

	\end{minipage}
\end{document}
